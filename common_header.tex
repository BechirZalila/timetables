% $Id$
\documentclass[a4paper]{article}
\usepackage{fancyhdr,graphicx}
\usepackage[a4paper,margin=1.5cm]{geometry}
\usepackage{array}
\usepackage{multicol}
\usepackage{makecell}
\usepackage{pdflscape}
\usepackage{diagbox}
\usepackage{color}
\usepackage{url}
\usepackage[table]{xcolor}

\usepackage[T1]{fontenc}
\usepackage{lmodern}
\renewcommand{\familydefault}{\sfdefault}

% Information diverses communes
% $Id$

\newcommand{\copyrightstart}{2014}
\newcommand{\copyrightend}{2020}
\newcommand{\copyrightholder}{Bechir ZALILA}
\newcommand{\copyrightholderemail}{\url{bechir.zalila@enis.tn}}


% Style de l'en-tête
\renewcommand{\headrulewidth}{0pt}
\lhead{\begin{picture}
    (0,0)\put(30,-60){\includegraphics[height=2cm]{ENIS_logo}}
  \end{picture}}
\rhead{\begin{picture}
    (0,0)\put(-120,-70){\includegraphics[height=2.5cm]{DGIMA_logo}}
  \end{picture}}

% Notice copyright
\newcommand{\footmessage}{\textcolor{purple!70!black!60!blue}{Emploi
    de Temps calculé avec FET \fetversion{} et mis en forme avec
    {\fontfamily{cmr}\selectfont \LaTeX{}} le \gendate{}. {\bf
      \copyright{} \copyrightstart{}-\copyrightend{}
      \copyrightholder{} (\copyrightholderemail{})}}}

% Format des titres de colonnes
\newcommand{\formathour}[1]{{\large\textbf{#1}}}
\newcommand{\hourone}  {\formathour{8:15  $\rightarrow$ 10:15}}
\newcommand{\hourtwo}  {\formathour{10:30 $\rightarrow$ 12:30}}
\newcommand{\hourthree}{\formathour{14:00 $\rightarrow$ 16:00}}
\newcommand{\hourfour} {\formathour{16:15 $\rightarrow$ 18:15}}

% Format des titres de lignes
\newcommand{\formatday}[1]{\large{\textbf{#1}}}
\newcommand{\lundi}   {\formatday{Lundi}}
\newcommand{\mardi}   {\formatday{Mardi}}
\newcommand{\mercredi}{\formatday{Mercredi}}
\newcommand{\jeudi}   {\formatday{Jeudi}}
\newcommand{\vendredi}{\formatday{Vendredi}}
\newcommand{\samedi}  {\formatday{Samedi}}

% Couleur d'une cellule vide
\definecolor{emptycellcolor}{RGB}{209, 209, 224}

% Format des séances hebdomadaires et des séances bi-mensuelles
\newcommand{\formatdh}[1]{\scriptsize{\makecell[{{m{5cm}}}]{\centering #1}}}
\newcommand{\formatdhh}[2]{
  {\diagbox[dir=NE,linewidth=1pt,linecolor=darkgray,width=5.4cm,height=1.9cm]
    {\scriptsize{\makecell[{{p{4.8cm}}}]{#1}}}
    {\scriptsize{\makecell[{{p{4.8cm}}}]{\raggedleft #2}}}}}

% Format des salles, matières, noms d'enseignants, nature de la séance
% et groupes d'étudiants.
\newcommand{\formatroom}[1]{\colorbox{black!70}{\textcolor{red!10}{\textbf{#1}}}}
\newcommand{\formatsubject}[1]{{\textbf{#1}}}
\newcommand{\formatteacher}[1]{\textsc{\tiny{#1}}}
\newcommand{\formattag}[1]{\textcolor{red!80!black}{\textbf{\emph{#1}}}}
\newcommand{\formatstudents}[1]{\underline{#1}}